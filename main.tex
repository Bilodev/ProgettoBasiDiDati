\documentclass{article}
\usepackage{tabularx}
\usepackage{graphicx}
\usepackage{longtable} % metti in preambolo
\usepackage{xcolor}
\usepackage{listings}



\author{Bilotta Antonio}
\title{Bookshelf Manager}
\begin{document}
\maketitle
\tableofcontents
\section{Idea del progetto}

Il concetto di \textit{possedere} una vasta collezione di libri per la maggior parte non letti è un qualcosa di già esplorato
da grandi autori quali \textbf{Umbero Eco}.\\ Una volta realizzato questo sistema, un utente più che possedere fisicamente dei titoli, avrà a disposizione un catalogo di volumi per tenere traccia di ciò che ha letto o meno.
Una piattaforma del genere vuole essere d'aiuto per quei tipi di persone, come il sottoscritto, che hanno bisogno di un "percorso" chiaro e ben definito, per
mantere e praticare con costanza un hobby quale può essere la lettura.

\section{Considerazioni sulla realtà d'interesse}
Ogni \textbf{Persona} sarà specializzata, con possibilità di overlapping, in \textbf{Utente}, o in \textbf{Admin}.\\
O gni \textbf{Utente} avrà delle informazioni anagrafiche legate al proprio profilo, come
\textit{nome, cognome, data di nascita}, oltre alla coppia \textit{email-password} che consentirà a questi
di effettuare il login.\\ Gli \textbf{Admin}, utenti speciali,
avranno accesso a delle aree speciali per caricare nuovi libri che saranno poi fruibili agli utenti.
\\\\
Ogni \textbf{Libro} avrà una sezione principale, che conserverà \textit{codice univoco, titolo originale, data prima edizione, voto medio}, a cui le varie \textbf{Edizioni},
contenenti \textit{titolo tradotto, lingua, immagine di copertina, numero pagine, data di pubblicazione} e una breve \textit{didascalia},
faranno riferimento, in modo che agli utenti un libro risulti letto in tutte le sue edizioni, nonostante abbiano usufruito solo di una.
\\Ogni \textbf{Edizione} avrà inoltre una \textbf{Prefazione}, composta da \textit{titolo e data} scritta
da un \textbf{Autore}.\\\\
Il sistema conserverà anche informazioni su \textbf{Autori}, che siano di libri o di prefazioni di questi, tra queste informazioni sulla vita quali
una piccola biografia, data di nascita e (in caso) di morte.
Ogni libro sarà \textit{classificato} in \textbf{categorie} diverse (conservate con multivalore), queste rappresentano generi e temi analizzati.
\\\\
Gli utenti, possono, dopo aver letto un libro, valutare la lettura, infatti effettueranno (non obbligatoriamente) delle \textbf{Recensioni}, divise in voto numerico
e commento, i libri così, potranno avere una valutazione generale data dalla media dei voti attribuiti dai lettori.
\\\\
Gli utenti avranno un numero totale di libri con il quale hanno interagito, poichè questi saranno divisi in due liste: libri letti, e da leggere
in futuro.
Per ogni utente sarà consevarto (opzionalmente) il libro che stanno attualmente leggendo.\\
Vogliamo memorizzare anche gli \textbf{Editori}, con \textit{codice, nome, nazione, sede} che stampano i vari \textbf{libri}.
\section{Specifiche Realtà d'interesse}
Per semplificare l'implementazione possiamo, successivamente alla Progettazione Concettuale, discutere alcuni
casi.
\\\\
Gli \textbf{Admin}, a livello implementativo, possono essere dei campi \textit{booleani} all'interno dell'entità \textbf{Utente}, vero se
all'utente sono concessi dei permessi speciali, falso altrimenti.\\\\
Inoltre le immagini dei libri voglio essere idealmente compresse e memorizzate fisicamente su un server con il quale l'applicazione
comunicherà, memorizzeremo quindi il \textit{percorso relativo} di queste all'interno del server che le memorizza, in particolare
il namespacing sarà il seguente: [\textit{codiceUnivicoLibro}]\textit{img.jpeg}.
\\\\
Le interazioni di ogni utente con i libri possono essere ridotte a un'interazione generica che conserva un flag di stato, 0 se letto, 1 se leggendo, 2 se finito.
questa conserva anche (in maniera opzionale) un riferimento ad una \textbf{Recensione} in caso il libro sia stato letto.

\section{Glossario dei termini}
\begin{center}
	\begin{tabularx}{1\textwidth} {
			| >{\raggedright\arraybackslash}X
			| >{\raggedright\arraybackslash}X |}
		\hline
		\textbf{Termine} & \textbf{Significato}                                                                     \\
		\hline
		Persona          & Insieme di dati anagrafici che qualificano un individo, che si specializza poi in utente \\

		\hline
		Admin            & Super utente che può aggiungere nuovi libri alla piattaforma                             \\
		\hline

		Utente           & Account di una persona che interagisce con libri e altri utenti                          \\
		\hline

		Autore           & Scrittore di libri o di Prefazioni di questi                                             \\
		\hline

		Prefazione       & Scritto, lungo o meno, che funge da premessa di un libro.                                \\
		\hline

		Libro            & Titolo originale, non fisico                                                             \\
		\hline

		Edizione         & Libro specifico, esiste fisicamente, poichè stampato da qualcuno                         \\
		\hline

		Recensione       & Pensiero e valutazione lasciato dagli utenti nei confronti di una lettura                \\
		\hline

		Editore          & Casa editrice che si occupa di redigere e mandare in stampa libri                        \\
		\hline
	\end{tabularx}
\end{center}

\newpage
\section{Progettazione Concettuale}
Procedendo con la progettazione concettuale della base di dati, si ottiene il
seguente \textbf{schema EER}:
\\
\begin{center}
	\includegraphics[width=1.1\textwidth]{img/BookshelfManagerEER.png}
\end{center}

\newpage
\subsection{Dizionario delle Entità}

Legenda: \colorbox{purple}{Sotto-entità}, \colorbox{yellow}{attributo multivalore},
\colorbox{cyan}{attributo ridondante}, \colorbox{green}{entità debole}, \colorbox{pink}{chiave candidata}
\\
\begin{center}
	\renewcommand{\arraystretch}{1.5} % più spazio verticale tra le righe  
	% --- Tabella Dizionario delle Entità ---
	\begin{longtable}{|p{2cm}|p{4cm}|p{3cm}|p{2.5cm}|}
		\hline
		\textbf{Termine}                & \textbf{Significato}                                           & \textbf{Attributi} & \textbf{Chiave} \\
		\hline
		\endfirsthead

		Persona                         & Individuo che interagisce con la piattaforma                   &
		\footnotesize
		\begin{itemize}
			\item ID
			\item Nome
			\item Cognome
			\item \colorbox{pink}{Email}
			\item Password
		\end{itemize}    & ID                                                                                                             \\
		\hline

		\colorbox{purple}{Utente}       & Account di una persona per interagire con libri e altri utenti &
		\begin{center}
			/
		\end{center}                  & \begin{center}
			                                /
		                                \end{center}                                                                                           \\
		\hline

		\colorbox{purple}{Admin}        & Utente con permessi speciali per caricare libri                &
		\begin{center}
			/
		\end{center}                  & \begin{center}
			                                /
		                                \end{center}                                                                                           \\
		\hline

		Libro                           & Titolo originale, concettuale                                  &
		\footnotesize
		\begin{itemize}
			\item ID
			\item titolo
			\item \colorbox{cyan}{voto}
		\end{itemize} & ID                                                                                                          \\
		\hline

		\colorbox{green}{Edizione}      & Versione fisica di un libro                                    &
		\footnotesize
		\begin{itemize}
			\item titolo
			\item lingua
			\item didascalia
			\item copertina
			\item numero pagine
			\item \colorbox{yellow}{generi}
		\end{itemize}             & \begin{center}
			                            /
		                            \end{center}                                                                                               \\
		\hline

		Autore                          & Scrittore di libri o prefazioni                                &
		\footnotesize
		\begin{itemize}
			\item ID
			\item nome
			\item cognome
			\item biografia
			\item data nascita
			\item data morte
		\end{itemize}              & ID                                                                                                         \\
		\hline

		Prefazione                      & Premessa di un libro                                           &
		\footnotesize
		\begin{itemize}
			\item ID
			\item titolo
		\end{itemize}                 & ID                                                                                                      \\
		\hline

		Lettura                         & Completamento di una edizione da parte di un utente            &
		\begin{center}
			/
		\end{center}                  & \begin{center}
			                                /
		                                \end{center}                                                                                           \\
		\hline

		\colorbox{green}{Recensione}    & Valutazione di un libro                                        &
		\footnotesize
		\begin{itemize}
			\item voto
			\item descrizione
		\end{itemize}               & \begin{center}
			                              /
		                              \end{center}                                                                                             \\
		\hline
		Editore                         & Casa editrice                                                  &
		\footnotesize
		\begin{itemize}
			\item ID
			\item nome
			\item nazione
			\item sede
		\end{itemize}                 & ID                                                                                                      \\
		\hline
	\end{longtable}
\end{center}


\subsection{Dizionario delle Relazioni}
\begin{center}
	\renewcommand{\arraystretch}{1.5} % più spazio verticale tra le righe  
	% --- Tabella Dizionario delle Entità ---
	\begin{longtable}{|p{2cm}|p{4cm}|p{3cm}|p{2	cm}|}
		\hline
		\textbf{Relazione}                        & \textbf{Descrizione}                                   & \textbf{Entità coinvolte} & \textbf{Attributi} \\
		\hline
		\endfirsthead


		Sta leggendo                              & Un utente è impegnato nella lettura di uno o più libri &
		Utente(0,N) \vphantom{2}Edizione(0,M)     & /                                                                                                       \\
		\hline

		Vuole leggere                             & Un utente salva una lista di libri da leggere          &
		Utente(0,N) \vphantom{2}Edizione(0,M)     & /                                                                                                       \\
		\hline

		Concludere                                & Un Utente ha concluso delle letture                    &
		Utente(0,N) \vphantom{2}Lettura(1,1)      & data fine                                                                                               \\
		\hline

		Appartenere                               & Una recensione appartiene a una lettura                &
		Recensione(1,1) \vphantom{2}Lettura(0,1)  & /                                                                                                       \\
		\hline

		Riferire                                  & Una lettura si riferisce ad un'edizione                &
		Lettura(1,1) \vphantom{2}Edizione(0,N)    & /                                                                                                       \\
		\hline

		Pubblicare                                & Un editore pubblica un'edizione di un libro            &
		Editore(1,N) \vphantom{2}Edizione(1,1)    & data                                                                                                    \\
		\hline

		RegistrareE                               & Admin registra un'edizione                             &
		Edizione(1,1)  \vphantom{2}Admin(0,N)     & data                                                                                                    \\
		\hline

		RegistrareL                               & Admin registra un libro                                &
		Libro(1,1)  \vphantom{2}Admin(0,N)        & data                                                                                                    \\
		\hline

		Stampare                                  & Un libro è stampato in diverse edizioni                &
		Edizione(1,1) \vphantom{2}Libro(0,N)      & /                                                                                                       \\
		\hline

		ScrivereL                                 & Un autore scrive un libro                              &
		Autore(1,N) \vphantom{2}Libro(1,1)        & data prima edizione                                                                                     \\
		\hline

		ScrivereP                                 & Un autore scrive una prefazione                        &
		Autore(0,N) \vphantom{2}Prefazione(1,1)   & data                                                                                                    \\
		\hline


		Anticipare                                & Una prefazione anticipa un'edizione di un libro        &
		Prefazione(1,1) \vphantom{2}Edizione(0,1) & data                                                                                                    \\
		\hline
	\end{longtable}
\end{center}

\subsection{Vincoli non esprimibili}
\footnotesize
\begin{itemize}
	\item L'attributo \textbf{voto} di un libro e delle recensioni deve essere un numero $n \pm 0.5$, con $0.5 \leq n \leq 9.5$
	\item L'attributo data di morte di un autore è opzionale.
	\item L'attributo email di un utente è univoco.
\end{itemize}
\section{Tavole}

\subsection{Tavole dei volumi}
È riportata di seguito la tavola dei volumi della base di dati

\begin{center}
	\renewcommand{\arraystretch}{1.5} % più spazio verticale tra le righe  
	% --- Tabella Dizionario delle Entità ---
	\begin{longtable}{|l|c|c|}
		\hline
		\textbf{Concetto} & \textbf{Tipo} & \textbf{Carico Applicativo} \\
		\hline
		\endfirsthead
		Persona           & E             & 100                         \\
		\hline
		Utente            & E             & 95                          \\
		\hline
		Admin             & E             & 5                           \\
		\hline
		Concludere        & R             & 300                         \\
		\hline
		Vuole leggere     & R             & 400                         \\
		\hline
		Sta Leggendo      & R             & 50                          \\
		\hline
		Lettura           & E             & 300                         \\
		\hline
		Appartenere       & R             & 150                         \\
		\hline
		Recensione        & E             & 150                         \\
		\hline
		Riferire          & R             & 250                         \\
		\hline
		Edizione          & E             & 300                         \\
		\hline
		RegistrareE       & R             & 300                         \\
		\hline
		RegistrareL       & R             & 100                         \\
		\hline
		Pubblicare        & R             & 100                         \\
		\hline
		Editore           & E             & 5                           \\
		\hline
		Stampare          & R             & 300                         \\
		\hline
		Libro             & E             & 100                         \\
		\hline
		ScrivereL         & R             & 100                         \\
		\hline
		ScrivereP         & R             & 30                          \\
		\hline
		Prefazione        & E             & 30                          \\
		\hline
		Autore            & E             & 20                          \\
		\hline
		Anticipare        & R             & 30                          \\
		\hline
	\end{longtable}
\end{center}

\subsection{Tavole delle operazioni}
		\renewcommand{\arraystretch}{1.5}
		\begin{tabular}{|l|c|c|}
			\hline
			\textbf{Operazione}                                                 & \textbf{Tipo} & \textbf{Frequenza} \\
			\hline
			Registrazione nuovo utente                                          & I             & 10/gg              \\ \hline
			Login utente                                                        & I             & 40/gg              \\ \hline
			Aggiunta libro da parte di un admin (RegistrareL)                   & I             & 2/gg               \\ \hline
			Aggiunta edizione da parte di un admin (RegistrareE)                & I             & 5/gg               \\\hline
			Inserimento di una lettura completata (Concludere)                  & I             & 15/gg              \\\hline
			Inserimento di una recensione associata a una lettura (Appartenere) & I             & 10/gg              \\\hline
			Aggiunta libro alla lista “Vuole leggere”                           & I             & 20/gg              \\\hline
			Aggiornamento stato lettura                                         & I             & 10/gg              \\\hline
			Generazione del voto medio di un libro                              & B             & 10/gg              \\\hline
			Import massivo nuovi editori                                        & B             & 1/settimana        \\
			\hline
		\end{tabular}


\section{Ristrutturazione}
\subsection{Analisi delle ridondanze}
Il dato ridondante è l'attributo "voto" dell'entità Libro.
Infatti sarebbe possibile ottenere la media dei voti attraversando le letture che fanno riferimento a tale libro, per accedere ai voti delle
recensioni, effettuando, poi, una media aritmetica. \\
Supponendo che l'attributo pesi \textbf{4 byte}, essendo un normale intero, e considerando il carico applicativo
dell'entità libro pari a 100 allora occupiamo circa \textbf{400 byte}.\\
Calcoliamo, con le tavole degli accessi se conviene mantenere o meno tale attributo.

\subsection{Tavole degli accessi}

Dalla tavola delle operazioni, risultano coinvolte nell'utilizzo del voto medio
solo le seguenti operazioni:

\vspace{1cm} % spazio sotto la linea
\noindent
\textbf{Operazione 6: Inserimento di una recensione (frequenza: 10/gg)}
\\\\
Ridondanza
\\\\
\begin{tabular}{|l|c|c|c|}
	\hline
	\textbf{Entità/Relazioni} & \textbf{Tipo} & \textbf{Accessi} & \textbf{Tipo di accesso} \\
	\hline
	Recensione                & E             & 1                & S                        \\
	Appartenere               & R             & 1                & S                        \\
	Lettura                   & E             & 1                & S                        \\
	Riferire                  & R             & 1                & S                        \\
	Edizione                  & E             & 3                & S                        \\
	Stampare                  & R             & 3                & S                        \\
	Libro                     & E             & 1                & L                        \\
	Libro                     & E             & 1                & S                        \\
	\hline
\end{tabular}
\\\\
$Totale = [2TOT_S + TOT_L] \cdot FREQUENZA =  [2\cdot 11+ 1] \cdot 10/gg = 230$ accessi/gg
\\\\\\No Ridondanza
\\\\\begin{tabular}{|l|c|c|c|}
	\hline
	\textbf{Entità/Relazioni} & \textbf{Tipo} & \textbf{Accessi} & \textbf{Tipo di accesso} \\
	\hline
	Recensione                & E             & 1                & S                        \\
	Appartenere               & R             & 1                & S                        \\
	\hline
\end{tabular}
\\\\
$Totale = [2TOT_S + TOT_L] \cdot FREQUENZA =  [2\cdot 2 + 0] \cdot 10/gg = 40$ accessi/gg

\vspace{0.5cm} % spazio sopra la linea
\hrule
\vspace{0.5cm} % spazio sotto la linea

\noindent
\textbf{Operazione 9: Generazione del voto medio di un libro (operazione batch, 10/gg) }
\\\\
Ridondanza
\\\\
\begin{tabular}{|l|c|c|c|}
	\hline
	\textbf{Entità/Relazioni} & \textbf{Tipo} & \textbf{Accessi} & \textbf{Tipo di accesso} \\
	\hline
	Libro                     & E             & 1                & L                        \\
	\hline
\end{tabular}
\\\\
$Totale = [2TOT_S + TOT_L] \cdot FREQUENZA =  [0 + 1] \cdot 10/gg = 10$ accessi/gg
\\\\\\No Ridondanza
\\\\\begin{tabular}{|l|c|c|c|}
	\hline
	\textbf{Entità/Relazioni} & \textbf{Tipo} & \textbf{Accessi} & \textbf{Tipo di accesso} \\
	\hline
	Libro                     & E             & 1                & L                        \\
	Stampare                  & R             & 3                & L                        \\
	Edizione                  & E             & 3                & L                        \\
	Riferire                  & R             & 1                & L                        \\
	Lettura                   & E             & 1                & L                        \\
	Appartenere               & R             & 1                & L                        \\
	Recensione                & E             & 1                & L                        \\
	\hline
\end{tabular}
\\\\
$Totale = [2TOT_S + TOT_L] \cdot FREQUENZA =  [0 + 11] \cdot 10/gg = 110$ accessi/gg

\subsubsection*{Conclusione}

Totale accessi con ridondanza $= (230+10) \ a/gg + 400byte$
\\
Totale accessi senza ridondanza $= (110+40) \ a/gg$
\\
Conviene eliminare il dato ridondante.

\subsection{Eliminazione Ridondanza}

\vspace{0.5cm}
\includegraphics[width=100px]{img/EliminazioneRidondanze.png}

\subsection{Eliminazione Gerarchie}
Lo schema inizialmente presenta la seguente specializzazione dell'entità
\textbf{Persona}.
\vspace{0.5cm}
\\
\vspace{0.5cm}
\includegraphics[width=130px]{img/EliminazioneGerarchie_Prima.png}
\\

\noindent
Procediamo sostituendo la gerarchia con due nuove relazioni che rendono le entità figlie due nuove entità deboli.
\\
\vspace{0.5cm}

\includegraphics[width=130px]{img/EliminazioneGerarchie.png}
\subsection{Eliminazione Attributi Multivalore}
\vspace{0.3cm}
\includegraphics[width=130px]{img/EliminazioneMultivalore_Prima.png}
\vspace{0.3cm}
\\
Sostituiamo all'attiributo multivalore \textbf{generi} l'entità \textbf{Genere} che sarà in relazione con un'\textbf{Edizione} con molteplicità $N:1$\\
\\
\includegraphics[width=120px]{img/EliminazioneMultivalore.png}
\\
\subsection{Schema EER ristrutturato}
\vspace{0.5cm}
\begin{center}
	\includegraphics[width=1.1\textwidth]{img/BookshelfManagerEER_Ristrutturato.png}
\end{center}
\section{Progettazione Logica}
\subsection{Schema relazionale}
Lo schema EER ristrutturato viene mappato nel seguente schema logico, implementiamo anche
ciò scritto nelle \textbf{Considerazioni sulla realta d'interesse}.
Rinominiamo le chiavi primarie del tipo ID in modo che siano nel formato [nomeEntità]ID, per una maggiore chiarezza.
\newline \newline
UTENTE(\underline{utenteID}, nome, cognome, email, password, isAdmin)\\\\
Libro(\underline{libroID}, titolo, voto, dataPrimaEdizione, autoreID$\uparrow$)\\\\
Genere(\underline{genereID}, genere)\\\\
Editore(\underline{editoreID}, nome, sede, nazione)\\\\
Edizione(\underline{edizioneID}, \underline{editoreID}$\uparrow$, \underline{libroID}$\uparrow$, titolo, lingua, didascalia, numeroPagine)\\\\
Autore(\underline{autoreID}, nome, cognome, biografia, dataNascita, dataMorte)\\\\
Prefazione(\underline{prefazioneID}, titoloEdizione, autoreID$\uparrow$)\\\\
CLASSIFICARE(\underline{EdizioneID}$\uparrow$, \underline{genereID}$\uparrow$)\\\\
LETTURA(\underline{utenteID}$\uparrow$, \underline{EdizioneID}$\uparrow$, status, dataFine, voto, mezzo, descrizione)\\\\
\noindent
\\
Abbiamo accorpato l'entià debole \textbf{Recensione} nella relazione \textbf{LETTURA} poichè le chiavi di queste due coincidono, inoltre con il flag status indichiamo se il libro è letto oppure se si sta leggendo, o si pianifica di leggerlo, tenendo, però a mente,
che queste ultime due opzioni implicano, che \textit{LETTURA.dataFine, LETTURA.voto, Lettura.mezzo, LETTURA.descrizione} siano NULL.
Notiamo che il campo voto sarà un intero da $0$ a $10$, se assume valore pari a $10$, allora il campo booleano mezzo, sarà $0$
poichè accettiamo mezzi voti (1.5, 3.5, 9.5, etc) solo se non eccedono il massimo, quindi  $10$
\subsection{Normalizzazione}
Lo schema relazionale rispetta le condizioni della \textbf{1NF}, poichè gli attributi di qualsiasi relazione sono atomici.
Sono rispettate anche le condizioni della \textbf{2NF}, poichè ogni attiributo di qualsiasi relazione non dipende 
parzialmente dalla chiave. \\
Lo schema è in \textbf{3NF} poichè non esistono dipendenze in cui attributi non primi (non parte di chiave) dipendono da attributi non chiave.\\
L'unica violazione della \textbf{BCNF} riguarda il campo \textit{Utente.email}, in quanto questo è univoco, quindi determina gli altri, però non è superchiave.
Decidiamo di \textbf{mantenere questa violazione}, per avere una maggiore coesione tra gli \textbf{ID} e per far sì che siano tutti numerici.

Di conseguenza, non ci sono ridondanze né anomalie di aggiornamento.
\section{Progettazione Fisica}
\subsection{Creazione delle tabelle}
\lstinputlisting[language=SQL]{creazioneTabelle.sql}
\subsection{Definizione delle operazioni}
\lstinputlisting[language=SQL]{operazioni.sql}
\section{Applicazione del Database}
\end{document}